\chapter{Discussion}\label{chap:discussion}

The main approach in this project, has been to divide the assignment into specialized manageable tasks, with well defined interfaces.  This has given each member of the group a chance to specialize in their given assignment area, without the need to worry about the larger picture of the project.



No practical application was chosen for the system and thus the scope of the project became to just build, test and document. This has been achieved and furthermore the system is in a quality and state to be reused and build upon either in seperate parts or as base for a pan-tilt application.



In contrast with previous projects, where the group have attempted to use responsibility based work models, specifically the Belbin model, this time focus was on expertice. Eeach individual group member primarily focused on their own practical field, and all other assignments were handled as they came naturally. 



Instead of having a lot of meetings, the group used a central log. This was an open Google document, where most of the  group communication took place. This freed up a lot of time, that would otherwise have been spent on meetings for the sake of coordination. 



The chosen way to manage the project meant that several problems were handled parallel. At some point, the conclusion of the dynamics chapter was to not implement state-space control, leading to other tasks like measuring velocity, becoming obsolete



It was agreed that the microprocessors software should be low coupled, but due to the parallel way of working this resulted in two different interpretations. One part of the software had low coupling to the operating system, which is absolutely viable for the driver modules that may well be used on other sytems. Another part of the software achieved low coupling with respect to the individual modules, which is also viable since the protocol developed can be used on any other media or the SPI connection can be used in other applications.



In the same way, the individual modules of the FPGA can be reused and implemented in other connections or build upon for specialized applications of the pan-tilt



\chapter{Conclusion}\label{chap:conclusion}

A mathematical model of the system was derived and analysed, and it was concluded that a state-space model of the system for simulation was excessive, so transfer functions was used for simulation instead. Simulink was used for simulating the PID-controller and parameters sufficient for the system, have been easily found this way. 



An interface for the hardware was developed, which lead to implementing PWM generators and counters for position on a FPGA using VHDL. This interface is working beautifully as a link between the ARM processor and the pan-tilt system using SPI communication.



On the ARM processor, running FreeRTOS, the control was implemented along with other tasks forming a user interface. Furthermore the software developed for the ARM processor is made so expansion of the system easily can be achieved.



A working system was developed throughout the project. The system is acting fast and furthermore it is robust with respect to the software developed. Even though some parts were downgraded to meet the deadline, the system performed well in test and is very precise.



The goals of the assignment has been met both on the tecnical level, group level and in individual specialization.