\chapter{Operating system}\label{chap:os}\marginnote{Frederik}
In the project-description it is stated that a microprocessor shall run a user interface, a regulation system and a SPI communication with the FPGA. This system will require the microprocessor to run multiple tasks and thus an operating system is necessary. It has been decided to use the lent ARM microprocessor so the operating system has to be available for this system.

The system has to run in real time to create a reliable environment for the regulation. Thus a RTOS have to be used.

The group have decided to use FreeRTOS both because it meets the requirements and is well documented, but also because the whole group have experience with FreeRTOS from the EMP course.

\section{ Pre-emptive Scheduling }

To secure that the specific tasks is run in real time pre-emptive scheduling will be implemented. This is implemented by giving tasks priority when they are declared. Thus all tasks which have to run in real time is given a high priority. If a task is scheduled to run, but a task is already running, new task has a higher priority it can temporarily stop the task. Run it's own task and resume the suspended task. That way a high priority task can be guaranteed to run at specific frequency or at certain interrupts.

When every task has been set up, the function \texttt{vTaskStartScheduler()} will initiate 


\section{ Delays  }

In the system tasks does either need to be blocked by a timer or an i/o port. In FreeRTOS the time delay is handled by to different calls. 
To secure that tasks run regularly each task is blocked with either \texttt{vTaskDelay} or \texttt{vTaskDelayUntil}. These calls will insure a relatively or a precise frequency of operation respectively. \texttt{vTaskDelayUntil} is able to do this by the user saving current tick at resume-time and giving that as input to the function.
 
\section{ Semaphores }

To synchronize processes and avoid race conditions, FreeRTOS use semaphores. The standard semaphore for synchronization between tasks is a binary semaphore. In FreeRTOS semaphores is created by, like with tasks first creating a handler, using the function \texttt{( xSemaphoreHandle xSemaphore )}

The FreeRTOS semaphore have two operations just like standard semaphores, the \texttt{wait()} and \texttt{signal()}. They are designed as \texttt{xSemaphoreTake} and \texttt{xSemaphoreGive}, respectively. The \texttt{xSemaphoreTake} checks if the semaphore is free and takes it if possible.
When obtained a semaphore can be released with \texttt{xSemaphoreGive}.

\section{ Queues }

In FreeRTOS queues are implemented as a mean for intertask communication. It works as an FIFO buffer where the different tasks can either push or pop from the buffer.

\section{ Suspend }
It is also possible to suspend and resume tasks by function calls. If a task is suspended it will not get any CPU time independent of it's priority.

\section{Risks of faults}

FreeRTOS does not do any work for error prevention. Does it is up to the developer to 

Is there a possibility of 
Data corruption
Deadlocks
Priority inversion



\section{Inter task communication}\marginnote{Leon}
When multiple cooperating tasks run under the same system, some service must be
provided for sharing information and data. There are two ways of doing this. One
is to allow access to some shared memory and the other is to provide message
passing mechanisms.

The main difference between the to ways is that while message passing must be
provided by the operating system, shared memory is entirely up to the programmer
to control. In either case race conditions must be considered.

Free RTOS provides queues and semaphores for inter task communication (Frederik
uddyber\ldots)

\subsection{Requirements}
In the considered pan-tilt system there are several paths requiring inter task
communication. By default the shared data must be protected by mutual exclusion
though it might be considered if data has to be reliable at all times.

 It must be taken into consideration that some of the time critical tasks needs
 to access multiple parameters. For example the control task needs access to all
 system parameters to calculate the control signals.
 
 The inter task communication system should ensure low coupling to allow reuse
 of sourcecode in other projects. It should also be considered, that the code
 might run on other operating systems so the coupling to OS should also be kept
 low.
 
 Since multiple programmers will have to use the inter task communication
 system, it should be considered to uphold some similarity between the
 interfaces though different data structures lies behind.
 
 The system must provide interfaces for parameters, events, states, queues
and counters. Parameters will define the current state of the external system
thereby representing the entire interface to the physical system. States will be
used to control the state of the internal system state machine.

Events are used for presenting user inputs and timing events to the tasks and
are signified in the message being deleted once read. Counters are also simple
data structures and is in essence just another way of interfacing to a global
variable. The reason for maintaining counters as a seperate interface is that
some systems provide hardware counters.

Though advanced queues are provided by freRTOS, a simple version is justified by
remaining a low coupling between the peripheral modules and the operating
system.

\subsection{Discussion}
A common way of sharing variables between tasks is to declare them globally and
then letting the individual tasks obtain a mutex and access the variable either
directly or by calling a function providing some interface to the variable.

The upside of this solution is that it is easily implemented. The downside is
increased coupling between different parts of the code, since multiple tasks are
depending upon knowledge of both the operating system to obtain the mutex and
the methods to access the variables.

Another solution would be to rely only on the queues and semaphores provided by
freeRTOS. This resembles message passing, since the interface is provided by the
operating system. The upside of this solution is, that coupling is now only
between the task and the operating system and could be further reduced by using
the preprocessor to provide a low coupling interface to any queue or semaphore
in any operating system. The downside is a considerable amount of overhead,
since the queue structure is a poor way of storing variables.

\subsection{Implementation}
The chosen solution is to build a simple common interface to shared memory. It
facilitates the simplicity of shared memory, while still retaining a low
coupling to the operating system. Another upside is that access to multiple
variables can be done under the same mutex, reducing overhead.

Seen from the calling application, variables are accessed by a function call.
Obtaining the mutex is done inside the function so instead of having a lot of
calls to the operating system sprinkled all over the source code, it is held in
one place. The interface function is based on a switch-case structure, ensuring
that as short time as possible is spend under the mutex.



