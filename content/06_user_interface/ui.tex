\chapter{User interface}\label{chap:ui}

\section{UI}
This chapter concerns building an application provideing a user interface to the
pan-tilt system. The application runs on the ARM Cortex M3 based EMP-board
(Reference ???)

\subsection{Requirements}
Functionality The UI must allow the user to see a numerical expression
describing the position of the system. The user must also be able to set the
wanted position of the system, either by entering the wanted values, or by
recording multiple positions and shifting between them.

It should also be considered making a simple solution where the potmeter
controls the pan and th digiswitch controls the tilt to provide some kind of
realtime control.

The main application for the UI will be in testing the pan-tilt system.
Therefore a logging system must be built to provide data for sthe development
and testing of the system. It is not definite what numbers and values are
interesting, so a versatile and easily adaptable system is required.

Though the system will be used mainly by people with some programming
experience, it must provide an easy way of working with the system code.
During testing there should be little focus on configuring and deep knowledge of
the source code should not be required.

\subsection{Discussion}
The EMP-board provides several channels for outputting data to the user. One is
by communication to a computer terminal. This is a good way of displaying large
amount of data, but should be kept in simple ASCII based text format. It is
imortant to provide a simple interface so output can be easily changed.

Another output channel is the two displays. Since focus is on displaying facts
rather than graphics, the small LCD display should not be used. Since the 16x2
is limited to displaying 32 characters at any time, it is necessary to implemet
some form of menu system.
 
The EMP board also provides a number of inputs. There are several buttons, but
given the easy to use requirements the use of buttons is avoided. That leaves
the digiswitch as the main input to control the menus and the potentiometer as a
secondary input.

The entire system should be seen as a big state machine controlling smaller
state machines so spltting the system into smaller controllable parts should be
considered

\subsection{Implementation}
Implementing a menu based state machine can be split into two parts. One
concerns the user navigating the menus and the other concerns having the menus
control the rest of the system.

Two ways: atomic system with many tasks blocking on queues or central task
executing functionality based on state of the system.

pros and cons: atomic system has overhead due to many tasks and many queues.
central system polling on data leads to inefficient use of CPU.
