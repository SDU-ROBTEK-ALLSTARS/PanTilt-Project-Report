\chapter{User interface}\label{chap:ui}
Considerations for building a user interface for a pan tilt system.

Platform The user interface (here after UI) must be implemented on a
demonstration board for the stellaris L3S696 (here after EMP board)

Functionality The UI must allow the user to see a numerical expression of the
position of the system. The user must also be able to set the wanted position of
the system, either by entering the wanted values, or by recording multiple
positions and shifting between them. It could also be considered making a simple
solution where the digiswitch and potmeter controls for setting the parameters
respectively, giving the user a kind of realtime control.

The main application for the UI will be in testing the pan-tilt system. It is
expected that collection and processing of data will be the responsibility of
the UI. Like in any testing it is not definite what numbers and values are
interesting, so a versatile and easily adaptable system is required.

Users Though the system will be used mainly by people with som programming
experience, it must provide a user friendly way of working with the system code.
During testing there should be little focus on configuring and deep knowledge of
the source code should not be required.

