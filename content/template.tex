\chapter{Example chapter}
Below follows examples of different \LaTeX commands
\section{Example code listing}
Cpp is the default language markup used


% Definition of a abbreviation. This will not show up in the text except in the "Abbreviations list" chapter
\nomenclature{ADHD}{Attention Deficit Hyperactivity Disorder}


\begin{lstlisting}[language={[ANSI]C++},caption={Example code},label={lst:examplecode}]
// Declare callback class
class MyPort1 : public DtmfCallback
{
	virtual void callbackMethod(DtmfInMessage * message)
	{
    // This function is called everytime data arrives. The incoming message only exist inside this method, so it is recommended to copy all needed information before exiting the function.
    // Fetch data from DTMF class to local memory (Assuming we have a dataContainer available in the MyPort1 class)
    m.getData(this->dataContainer*, 0, m.getMessageLength());

     // Data is now stored locally, return.
    return;
	}
}
// Make an instance of the class
MyPort1 * myPort1 = new MyPort1();
// Add the object as callback for port 1.
api->servicePort('1',myPort1);
\end{lstlisting}



\subsection{Example table}

\begin{table}[htb]								%[htb] means here, top or bottom (where latex tries to place the float on the page)
	\centering
	\begin{tabular}{c|c|c}					% | is a vertical line ; c means center, l left and r right
	Byte 1 & Byte 2 & Byte 3 \\			% the &-sign seperates columns
	\hline													%horizontal line
	\verb@TYPE@ & \verb@KOMMANDO@ & \verb@DATA@ \\
	\end{tabular}
	\caption{Telegramformat}				% The caption (billedtekst!). All floats should have a caption
	\label{tab:telegramformat}			% reference ID. Tables are of the format \label{tab:*}
\end{table}






\subsection{Example figure}

\begin{figure}[htb]
	\centering
	\includegraphics[scale=1,trim=0 0 0 0]{content/00_frontmatter/sdu_logo.pdf} %trim=l b r t (can cut off from every side)
	\caption{SDU logo}
	\label{fig:sdu_logo_xx}			% figure labels are of the form \label{fig:*}
\end{figure}






\subsection{Example equations}

\begin{equation}						% one-line equation
	y''(t)+2y'(t)-3y(t)=2t
	\label{eq:opg1}
\end{equation}


\begin{align}								% equation on several lines, aligned where the &-sign is
	y(0)&=1\notag\\						% notag means this line is not numbered
	y'(0)&=2
	\label{eq:opg1_beting}		% equation labels are of the form \label{eq:*}
\end{align}

\begin{equation}
	\boxed
	{
		\mathcal{L}\{f^{(n)}\} = s^{n} \mathcal{L}\{f\} - \sum_{i=1}^{n} s^{(n-i)} f^{(i-1)}(0)		%mathcal = math calligraphy
	}
	\label{eq:opg1_lapl_difflign}
\end{equation}

																	% \footcite[224]{kreyzig} = Reference to kreyzig side 224 in a footnote
																	% \eqref{} = Reference to an equation (or \align)

\begin{align*}			% the star means this whole \align section is un-numbered
	\mathcal{L}\{y''(t)\}+2\mathcal{L}\{y'(t)\}-3\mathcal{L}\{y(t)\}&=\mathcal{L}\{2t\}\\
	&\implies \\
	s^{2} Y(s) - s y(0) - y'(0) +2 \left( s Y(s)-y(0) \right) -3 Y(s) &= \dfrac{2}{s^{2}}
\end{align*}





\subsection{Bullets}

\begin{itemize}
	\item	One
	\item	Two databehandlingstid.
	\item	Three
\end{itemize}




\subsection{Numbering}

\begin{enumerate}
	\item This is the first option
	\item and this is the second
\end{enumerate}
