\chapter{Control}\label{chap:llc}

This chapter concerns implementation of a regulator for the pan tilt system.

\section{requirements}

\section{implementation}
The control algorithm is implemented in the control task. All parameters are provided from the parameter server but are cast as single precision floating point values for exact calculations.

When run the control task saves the value of the tick counter and when finished, it uses the FreeRTOS vTaskDelayUntil API to calculate when to unblock. As a starting point the control task runs at one hundred hertz frequency.

To calculate the error, a common format is needed. The parameter server holds the setpoint in degrees and the actual position in ticks. To be able to have an intuition about the control algorithm, degrees are cosen and the position in degrees is calculated. The error is then calculated as the difference between setpoint and actual position and is thus a signed value in degrees.

\subsection{Implementing proportionality gain}
A first guess for the proportionality gain is to make an error of ten degrees resemble a ten percent PWM signal and an error of a hundred degrees resemble full speed. The PWM signal range from five thousand, where the motor barely moves up to the maximum of a 16 bit signed, where the motor runs full speed.
\begin{equation}
2^{15} = 32.767 \Rightarrow 
32.767 - 5000 = 27.767 \Rightarrow 
1 \% = \frac{27.767}{100} \approx  277
	\label{eq:PWM}
\end{equation}
 Therefore the first guess is calculated:
\begin{equation}
10x = 5000 + (10 * 270) => x = \frac{7770}{10} \approx  8
	\label{eq:P-term}
\end{equation}
Since this makes it possible to get PWM values out of the range, a maximum function is implemented, so that absolute values out of range are corrected to the maximum value.

Since PWM values below a certain minimum does not make the motor move, a bias is added so that non-zero values are added with the bias value. A goal area is implemented as so that all values inside the area is zeroed before the bias is added. If this was not done even the smallest error would be biased.

\subsection{Adjusting the P-gain}
Testing of the system confirms that the proportionality factor works well and the system tracks the given position. Testing on only the tilt part of the system shows, that at proportionality gain values above 22, the system becomes unstable. The maximum stable value is at 16 but this makes the system less robust, resulting in instability when the system is stressed, so a value around 10 seems fit.

\subsection{Integrator}
Since the error grows smaller as the goal is approached, movement almost halt when approaching the goal area, making the system less accurate and slower. Therefore an integration term is implemented by adding the errror to the integration value on each run. Thereby in the situation where the error is too small to make the system move, the integration term will rise and thus add to the input. 

Normally an integration would mean the product of the value and the time since last sample, but presuming that this time is nearly constant it can be calculated as part of the I-term. 

To keep the integration from going to infinity, an anti wind up filter is implemented. Wind up happens for example when the system is stopped while not at the setpoint and the integration reaches high values that has to be overcome when the system resumes.

\subsection{Adjusting the I-gain}
Testing was done with a fixed P-gain of ten and integrator windup max at ten thousand. As a first guess the I-gain was set to 1 which made the system unstable. Additional testing showed that values a hundred times less were more appropriate. The system becomes unstable around 0.1 and a value of 0.05 was chosen.

Testing showed that the tracking ability of the system had substantially increased and the speed has increased especially when close to the goal. Only after changing setpoint, the system seems slow. This suggest implementing a differentiator term.

\subsection{differentiator}
By saving the error at each entry it is possible to calculate the change in the error. This term will be particularly large just after changing the setpoint, and can thus help improve the acceleration.

The actual derivative would be the change in value over the time since last entry, but as for the integrator, this can be omitted presuming that it is constant.

In a faulty system it could be considered implementing a maximum on the differentiator to prevent wrong or noisy measurements of having too much impact on the system. As a start this is not implemented as to not solve the problem until it arise.

\subsection{Adjusting the D-gain}
Testing was done with a fixed P-gain of ten, I-gain of 0.05 with anti-windup at ten thousand. As a first guess the D-gain was set to 0.1 which made the system unstable. Further testing showed unstability starting around 0.08 and 0.05 was chosen.

The system reacts faster when changing setpoint, but the D-term introduces some jitter when on hold. Though the system seems less robust, it reacts fine to applied external force.




























