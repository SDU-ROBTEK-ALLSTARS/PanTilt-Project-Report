\section{LLC testing}
Each subsytem in the low level control section is encapsulated in a seperate module.
Each module is tested seperately before being assembled in the main memory interface block, and this chapter will deal with the individual module tests, as well as the final test.
The test consists of a testbench file and an associated evaluation.

\subsection{PWM generator}
Range:                  PWM Generator (PWMv1.vhd)
Testbench: 				pwmtester.vhd
Search area:            Can a duty cycle be set, and will it produce a corresponding correct output?
						If the input is all zero or max value, will the module output dc ?
Expectations:           The expectations are that all of the above will work.
Result:                 It worked
Conclusion:             The FPGA can control the motor. 

\subsection{Signed to magnitude}
Range:                  Signed int to magnitude and sign splitter (SignedToMagnitude.vhd)
Testbench: 				signtest.vhd
Search area:            Every possible(18 bit) integer is input, and the output is examined.
Expectations:           That the magnitude of the number is on the output pins, as well as the sign.
Result:                 The correct results for all integers, except for the most negative, which was truncated to maximum magnitude.
Conclusion:             The splitter works as intended, and can be used to direct the duty cycle input. 

\subsection{Controlblock}
Range:                  Top motor control module (ControlBlockTop.vhd)
Testbench: 				motor\_control\_block\_test.vhd
Search area:            Can the module generate a PWM signal based on a signed integer, switch to freerun and brake ?
Expectations:           It is expected that the module works as described.
Result:                 The module works
Conclusion:             The pan/tilt system can be convieniently controlled by a signed integer and a control bit.

\subsection{Input filter}
Range:                  Input filter(Hysteresis.vhd)
Testbench: 				inputfiltertest.vhd
Search area:            Does the input filter, filter out high frequency noise.
Expectations:           It filters out noise and the signal had a slight delay.
Result:                 Noise was removed and the signal was delayed by a couple of nano seconds.
Conclusion:             Disturbances was removed from the signal reliably, and the signal can be used as decoder input.

\subsection{Quadrature decoder}
Range:                  Quadrature decoder module (FourXDecoderV2.vhd)
Testbench: 				decoder\_v2\_testb.vhd
Search area:            Can the decoder correctly keep track of a position, when a pule train is applied to the inputs.	
Expectations:           The decoder counts up/down with the given input signal.
Result:                 The decoder counted correctly.
Conclusion:             The decoder can be used to keep track of the pan/tilt position.

\subsection{Velocity estimator using secant method}
Range:                  Velocity approximations(dxdt)
Testbench: 				velocity\_test\_secant.vhd
Search area:            Can the velocity estimator \"guess\" the velocity, when applied a changing position.	
Expectations:           A reasonable velocity.
Result:                 Garbage output.
Conclusion:             It would have been nice if this module worked, but it is not strictly needed and therefore it does not matter that it did not work.

\subsection{velocity estimator avg}
Range:                  Velocity approximations(dvdtv2)
Testbench: 				velocity\_test\_avg.vhd
Search area:            Can the velocity estimator \"guess\" the velocity, when applied a changing position.
Expectations:           A reasonable velocity.
Result:                 Garbage output.
Conclusion:             It would have been nice if this module worked, but it is not strictly needed and therefore it does not matter that it did not work.

\subsection{Watchdog}
Range:                  (Which system part is tested?) module file
Testbench: 				watchdog\_test.vhd
Search area:            Can the watchdog maintain an \"alive\" signal when an alternating input is applied, and will the output be low when the input dies.
Expectations:           The above.
Result:                 It works perfectly, the output dies a short time after the input.
Conclusion:             The watchdog can be used in the project.
