\section{Performance testing}

\subsection{Precision of the system}\label{subsec:precisionofsystem}
This experiment will test the precision of the complete system with the
parameters found from simulating the system. The parameters can be found in
table \ref{tab:actual_gain_values}

\subsection*{Setup}

The test is performed by fastening a laser pointer on the pan/tilt system. A
board is placed next to the system and two positions where the pointer is on the
board are chosen as seen in figure \ref{fig:systemtestsetup}. The laser pointer is placed 180 cm
from the board. Thus corresponds to 3cm on the board.

\[ \tan(1 \ deg) \cdot 180 \ cm \approx 3,14 \ cm \]


\begin{figure}[htb] \centering \includegraphics[width=\textwidth,trim=0 0 0
0]{graphics/overallsystemtest.png} %trim=l b r t (can cut off from every side)
	\caption{Setup of the test. From left to right; the mounted laser, the system pointing at the board, laser dot and marks.}
	\label{fig:systemtestsetup}			% figure labels are of the form \label{fig:*}
\end{figure}

The system runs in automode and changes between two setpoints. The positions
are marked and the test ends after ten iterations. 

\subsection*{Results}

At both points the error was less than 5 cm and most of the time
less than two centimetres. This means that the system have an accuracy between 0.658 and
3.293 degrees.

\[ \frac{2,0 \ cm}{3,14\ cm/deg} = 0,64 \ deg \]

\[ \frac{10,0 \ cm}{3,14 \ cm/deg} = 3,19 \ deg \]

\subsection*{Conclusion}

The system itself, has a precision of one third degree.
\[ \frac{360 deg}{1080 tick} = 1/3 deg/tick \]

It is a high precision that have been observed, at times the uncertainty is as
small as the precision of the system. Though it does also show that a higher
precision can be obtained. Overshooting was though occurring in the control.
Additional tuning of the parameters might enhance the performance.

\subsection{Precision of new parameters}\label{sec:precisionofsystem2}

This experiment will test the PID values found in table \ref{tab:actual_gain_values}
under best performance.

Here the system will run in automode between two position but only one of the
positions are marked. 

\subsection*{Setup}

The setup is similar to the setup in section \ref{subsec:precisionofsystem}, see
Figure \ref{fig:systemtestsetup}, with the exception that only one point is measured and the distance to the board is increased to 370 cm. Thus each degree compares to 6,46 cm. This is repeated ten times.

\[ \tan(1 \ deg) \cdot 370 \ cm \approx 6,46 \ cm \]

\subsection*{Results}

In this test the precision of the pan and tilt were recorded to be different. The vertical precision was 7,5 cm, but the horizontal precision was just 2,0 cm. Thus the pan/tilt system have
reached precision of respectively 0.31 degrees and 1.16 degrees.


\[ \frac{2,0}{6,46} = 0.31 \]

\[ \frac{7,5}{6,46} = 1.16 \]

\subsection*{Conclusion}
The pan has reached a constant precision on par with the system precision
of one third of a centimeter. The tilt part has also become adequately precise, but
it should be possible to make it perform even better. It could be that because the
tilt is affected by gravity it is harder to control than the pan.


