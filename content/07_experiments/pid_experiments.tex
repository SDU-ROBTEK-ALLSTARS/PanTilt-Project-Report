\chapter{PID experiments}\label{sec:pid_experiments}
Since adjusting the PID gain values is a somewhat empiric process, the
observations are described in the following as multiple tests although they were
done consecutive.

\section{Adjusting the P-gain}\label{sec:pid_experiments_p}
The purpose of this test is to find suitable values for the P-gain and to verify
the values found in simulation.

\section*{Setup} The test was performed using the setup described in
section \ref{subsec:precisionofsystem}. The system is given two positions and runs in
automode between them. First the P-gain was increased until the system became
unstable, in this test meaning unable to come to rest at the given position.

\section*{Results} From equation \ref{eq:conv} the gains found in simulation can be
converted to the actual system and be compared to the results. The converted
values can be seen in table \ref{tab:actual_gain_values}. At P-gain of 600 the gear
belts of the system starts to notch over teeth, while still being stable, so
testing at higher gains were omitted.

It was observed that the system has difficulties reaching
the goal position. This is due to the error becoming smaller and smaller on
approaching the position. The P-gain values found in simulation worked for the
system.

\section*{Conclusion} Testing of the system confirms that the control
algorithm works well and the system tracks the given position. The difficulties
reaching the position suggests an I-term should be added.

\section{Adjusting the I-gain}\label{sec:pid_experiments_i}
The purpose of this test is to find suitable values for the I-gain and to verify
the values found in simulation.

\section*{Setup} The test was performed using the setup described in
section \ref{subsec:precisionofsystem}. The system is given two positions and runs in
automode between them. During the test the P-gain was kept at 20.

First the I-gain was increased until the system became
unstable, in this test meaning unable to come to rest at the given position.

From equation \ref{eq:conv} the gains found in simulation can be converted to the actual
system and be compared to the results. The converted values can be seen in
table \ref{tab:actual_gain_values}. 

\section*{Results} 
At I-gain of 4 the gear belts of the system starts
to notch over teeth, while still being stable, so testing at higher gains were
omitted.

The sytem reaches the position, but after the overshoot,
especially the tilt halts for a while. This is due to the integrator being at
minimum when passing the setpoint and therefore it takes a while before the
positive error is summed up to a positive contribution to the input. This
suggests adding a D-term at least to the tilt. The problem is insignificant on
the pan due to the relatively small errors compared to the tilt.

\section*{Conclusion} Testing showed that the tracking ability of the system
had substantially increased and the speed has increased especially when close to
the goal. The values found from autotune in the simulation works well, but the
system seems to react fine with even higher values, but a  D-term should be
added to get faster recovery after overshoot.

\section{Adjusting the D-gain}\label{sec:pid_experiments_d}
The purpose of this test is to find suitable values for the D-gain and to verify
the values found in simulation.

\section*{Setup}
The test was performed using the setup described in
section \ref{subsec:precisionofsystem}. The system is given two positions and runs in automode between them. During the test
the P-gain was kept at 20 and the I-gain at 2.

First the D-gain was increased until the system became unstable, in this test
meaning unable to come to rest at the given position.

From equation \ref{eq:conv} the gains found in simulation can be converted to the actual
system and be compared to the results. The converted values can be seen in
table \ref{tab:actual_gain_values}. 

\section*{Results}
At D-gain of 90 the system becomes unstable.

The D-gain values obtained from simulation worked fine, but it takes a couple of
seconds to finetune into position.

\section*{Conclusion}
The PID-regulator now works well and is able to reach a given position. Some
fine-tuning is needed to get perfect parameters.

\begin{table}[htb]
	\centering
	\begin{tabular}{lccc}			
	Term & Simulation autotune & Measured maximum & Best performance  \\
	\midrule								
	P-gain pan & 36,25 & 600 & 40 \\
	P-gain tilt   & 18,50 & 600 & 20 \\
	I-gain pan  & 0,78 & 4 & 1,5 \\
	I-gain tilt   & 0,40  & 4 & 0,5 \\
	D-gain pan & 78,25 & 90 & 0,5 \\
	D-gain tilt   & 40,50 & 90 & 2 \\
	\end{tabular}
	\caption{P-gain values converted to be used in the control algorithm and the values measured on the system.}				
	\label{tab:actual_gain_values}			
\end{table}

\section{Tuning the PID parameters}
The purpose of this test is to increase the accuracy of the system by changing
the drive belt and retune the parameters of the PID.

\section*{Setup}
The test was performed using the setup described in section \ref{subsec:precisionofsystem}. The
system is given two positions and runs in automode between them while tuning the
parameters of the PID.

\section*{Results}
The drive belt on the pan was replaced.

The PID parameters were tuned slightly, reducing the D-gain and increasing the
I-gain to obtain better precision. The new values can be found in
table \ref{tab:actual_gain_values} under best performance. Small changes were made to
parts of the control algorithm concerned with evaluating when the position is reached.

The precision of the system was raised to a level where this test can no
longer be used to measure the inaccuracy. 

A periodic error was observed as the small drive wheel on the motor sometimes
jumps a tooth and after a while jumps back. The belt seems fine, but
the teeth of the drive wheel is worn

\section*{Conclusion}
The performance of the system has been greatly improved and new tests should be
performed with greater distance between the pan/tilt and the board. 

If the error with the drive belt continues, the drive wheel should be replaced.


\section{Verification of the control frequency}\label{subsec:control_freq}
The purpose of this test is to verify that the control task actually runs at 100 Hz.

\section*{Setup}
At the entry of control task a pin is set, and at the end of the control task it
is cleared. An oscilloscope is connected to the pin. While the system is running, the period between rising edges is measured using the oscilloscope.

\section*{Results}
The results vary between 9,80 and 9,90 ms. This is equal to between 101,01 Hz and 102,04 Hz, meaning an error of less than 2 percent.

\section*{Conclusion}
The control task runs close to 100 Hz and therefor jitter should not be a problem.









