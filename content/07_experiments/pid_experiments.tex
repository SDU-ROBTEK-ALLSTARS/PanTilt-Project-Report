\section{PID experiments}\label{sec:pid_experiments}
Since adjusting the PID gain values is a somewhat empiric process, the observations are described in the following as multiple tests although they were done consecutive.

\section{Adjusting the P-gain}\label{sec:pid_experiments_p}
The purpose of this test is to find suitable values for the P-gain and to verify the values found in simulation.

\subsection{Setup}
The test was performed using the setup described in (REFERENCE TIL FREDERIKS OPSTILLING). The system is given two positions and runs in automode between them.

\subsection{Procedure}
First the P-gain was increased until the system became unstable, in this test meaning unable to come to rest at the given position.

\subsection{Results}
From \ref{eq:conv} the gains found in simulation can be converted to the actual system and be compared to the results. The converted values can be seen in \ref{tab:actual_gain_values}. At P-gain of 600 the gear belts of the system starts to notch over teeths, while still being stable, so testing at higher gains were omitted.

\subsection{Discussion}
It was observed that the system has difficulties reaching the goal position. This is due to the error becoming smaller and smaller on approaching the position. The P-gain values found in simulation worked for the system, but seemed low.

\subsection{Conclusion}
Testing of the system confirms that the control algorithm works well and the system tracks the given position. The difficulties reaching the position suggests an I-term should be added.

\subsection{Adjusting the I-gain}\label{sec:pid_experiments_i}
The purpose of this test is to find suitable values for the I-gain and to verify the values found in simulation.

\subsection{Setup}
The test was performed using the setup described in (REFERENCE TIL FREDERIKS OPSTILLING). The system is given two positions and runs in automode between them. During the test the P-gain was kept at 20.

\subsection{Procedure}
First the I-gain was increased until the system became unstable, in this test meaning unable to come to rest at the given position. 

From \ref{eq:conv} the gains found in simulation can be converted to the actual system and be compared to the results. The converted values can be seen in \ref{tab:actual_gain_values}. At I-gain of 4 the gear belts of the system starts to notch over teeths, while still being stable, so testing at higher gains were omitted.

\subsection{Discussion}
The sytem reaches the position, but after the overshoot, especially the tilt halts for a while. This is due to the integrator being at minimum when passing the setpoint and therefore it takes a while before the positive error is summed up to a positive contribution to the input. This suggests adding a D-term at least to the tilt. The problem is insignificant on the pan due to the relatively small errors compared to the tilt.

The values found from autotune in simulation made the system unstable. This was the case wether the values were used as is or were converted as suggested in \ref{sec:integrator}.

\subsection{Conclusion}
Testing showed that the tracking ability of the system had substantially increased and the speed has increased especially when close to the goal. Though the simulated values did not work, it was possible to find suitable values for the I-term. A D-term should be added to get faster recovery after overshoot.

\subsection{Adjusting the D-gain}\label{sec:pid_experiments_d}
The purpose of this test is to find suitable values for the D-gain and to verify the values found in simulation.

\subsection{Setup}
The test was performed using the setup described in (REFERENCE TIL FREDERIKS OPSTILLING). The system is given two positions and runs in automode between them. During the test the P-gain was kept at 20 and the I-gain at 2.

\subsection{Procedure}
First the D-gain was increased until the system became unstable, in this test meaning unable to come to rest at the given position. 

From \ref{eq:conv} the gains found in simulation can be converted to the actual system and be compared to the results. The converted values can be seen in \ref{tab:actual_gain_values}. At D-gain of 90 the system becomes unstable.

\subsection{Discussion}
The system reaches the position, but it takes a couple of seconds to finetune into position. The D-gain values obtained from simulation had little effect while used as is, but made the system unstable if converted as suggested in \ref{sec:integrator}.

\subsection{Conclusion}
The PID-regulator now works well and is able to reach a given position. Some fine-tuning is needed to get perfect parameters.

\begin{table}[htb]				
	\begin{center}
	\begin{tabular}{l|c|c|c|c}			
	Term & Simulated unstable & Simulated autotune & Measured unstable & Good performance  \\	\hline								P-gain pan & 112,5  & 4,25 & 600 & 20 \\
	P-gain tilt   & 112,5 & 9,20 & 600 & 20 \\
	I-gain pan & 62,5 & 2,79 & 4 & 2 \\
	I-gain tilt   & 87,5 & 5,33 & 4 & 2 \\
	D-gain pan & 17,5 & 0,21 & 90 & 1 \\
	D-gain tilt   & 25,0 & 0,24 & 90 & 1 \\
	\end{tabular}
	\end{center}
	\caption{P-gain values derived from testing}				
	\label{tab:actual_gain_values}			
\end{table}


